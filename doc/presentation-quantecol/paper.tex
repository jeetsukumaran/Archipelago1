\documentclass[11pt]{article}
\usepackage{amsmath}
\usepackage{geometry}                % See geometry.pdf to learn the layout options. There are lots.
\geometry{letterpaper}                   % ... or a4paper or a5paper or ... 
%\geometry{landscape}                % Activate for for rotated page geometry
%\usepackage[parfill]{parskip}    % Activate to begin paragraphs with an empty line rather than an indent
\usepackage{graphicx}
\usepackage{amssymb}
\usepackage{epstopdf}
\DeclareGraphicsRule{.tif}{png}{.png}{`convert #1 `dirname #1`/`basename #1 .tif`.png}

\title{Archipelago}
\author{Jeet Sukumaran}
%\date{}                                           % Activate to display a given date or no date

\begin{document}
\maketitle
\section*{Introduction}
Archipelago is a forward-time simulation of macro-evolutionary diversification processes in a spatially-explicit framework.
The motivation for this simulator is to provide for null distributions of diversity, i.e., numbers of species or lineages found in different regions, under the model being simulated.

\section*{The Simulation Model}

\subsection*{The Diversification Process}
The birth-death process has two parameters.
The birth rate, $\lambda$, is the probability that each species in the system speciates, or splits into two daughter species.
The death rate, $\mu$, is the probability that each species in the system goes extincts.
A special case of the birth-death process is when the death rate, $\mu$, is 0, in which case we have a pure-birth process, which is also referred to as the \textit{Yule} model.

In the pure birth process, the expected number of lineages, $E(n)$ at time $t$ is given by:
\begin{align*}
E(n) = e^{\lambda t},
\end{align*}
corresponding to population growth equations of the same form.

With extinction, the expected number of lineages, $E(n)$ at time $t$ is:
\begin{align*}
E(n) = e^{(\lambda-\mu) t}.
\end{align*}

\subsection*{The Geographic Template}

The spatial aspect of the simulation model is represented by the \textit{geographical template}, which the defines the fundamental atomic spatial units of the simulation, \textit{regions}, and the connectivity between these units, which determines the rate of dispersal of lineages from one region to another.





%\subsection{}



\end{document}  